\documentclass[preprint, aps, pra]{revtex4-1}

\usepackage{amsmath}
\usepackage{amssymb}
\usepackage{bm}
\usepackage{esint}

%\usepackage[default]{sourcesanspro}
%\usepackage[T1]{fontenc}

\newcommand{\ab}{{\alpha\beta}}
\newcommand{\eps}{\varepsilon}
\newcommand{\bx}{{\bm{x}}}
\newcommand{\bv}{{\bm{v}}}
\newcommand{\bu}{{\bm{u}}}
\newcommand{\bc}{{\bm{c}}}
\newcommand{\bg}{{\bm{g}}}
\newcommand{\bn}{{\bm{n}}}
\newcommand{\be}{{\bm{e}}}
\newcommand{\bw}{{\bm{w}}}
\newcommand{\pd}{\partial}
\newcommand{\kp}{\kappa}

\begin{document}
  \title{Sheared Granular Gas Dynamics}
  \author{Yernur Baibolatov and Frank Spahn}
  \affiliation{Universit\"{a}t Potsdam, Institut f\"{u}r Physik und Astronomie}
  %\maketitle
  
\section{Non-uniform granular gas system}
 In this work we consider a rotating disk of granular gases with non-uniform size distribution of constituents
 under external gravitational shear. Granular gases are well known for its intrinsic dissipative nature and hence
 decay of velocity dispersion or \emph{granular temperature} if the system has nonzero initial energy. Let us start
 our work with notations and definitions of necessary parameters. The total number of constituents in the system is $N$.
 The number of species in our system is $s$ and the total number of constituents of species $\alpha$ is $N_\alpha$, 
 where $\alpha\in[1, s]$.
 Hence, we can write
 \begin{equation}
   \chi_\alpha=\frac{N_\alpha}{N}\;,\,\,\,\sum_{\alpha=1}^{s}\chi_\alpha=1\;,
 \end{equation}
 where $\chi_\alpha$ is the concentration of constituents of species $\alpha$.
This means that there are $N_\alpha$ identical constituents with masses $m_\alpha$. 
Obviously $m_\alpha\neq m_\beta$ if $\alpha\neq\beta$. Since a system of certain species consists of large number
of constituents $N_\alpha$, it is considered as a statistical system and we analyze it through certain macroscopic parameters.
Let us say that $P_\alpha$ is one of the macroscopic parameters of species $\alpha$, then for the whole system we can write
\begin{equation}\label{eq:macroparameter_discrete}
  P = \sum_{\alpha=1}^{s}\chi_\alpha\cdot P_\alpha\;.
\end{equation}
This is the mean value of macroparameter for the whole system across all species.

In order to define the macroparameter $P_\alpha$ itself, we need to introduce the one-particle distribution function in the phase
space of dynamic variables. The only dynamic variables of a single particle are coordinate $\bx$ and velocity $\bv$.
Now, the one-particle distribution function, or simply distribution function, for species $\alpha$ is written as $F_{\alpha}(\bx,\bv)$.
This function should have the next property
\begin{equation}
  \int F_\alpha(\bx,\bv)d\bx d\bv = n_\alpha = \frac{N_\alpha}{V}\;,
\end{equation}
where the integration is performed over all phase space and $n_\alpha$ is the number density of species $\alpha$, $V$ is the total volume 
of the system. Note that $n_\alpha$ here is the average number density across the whole system volume $V$. We can write separately 
coordinate and velocity distribution functions as 
\begin{equation}
  \begin{split}
    g_\alpha(\bx) &= \int F_\alpha(\bx,\bv)d\bv\;,\\
    f_\alpha(\bv) &= \int F_\alpha(\bx,\bv)d\bx\;,
  \end{split}
\end{equation}
and obviously
\begin{equation}
  \int g_\alpha(\bx)d\bx = \int f_\alpha(\bv)d\bv = n_\alpha\;.
\end{equation}
Further, we will mostly use the velocity distribution function, since we know that all informative
macroscopic parameters describing statistical systems are certain moments of dynamic functions of velocity $p_\alpha=p_\alpha(\bv)$.
Hence corresponding macroparameter $P_\alpha$ is obtained from
\begin{equation}
  n_\alpha P_\alpha = \int p_\alpha(\bv)f_\alpha(\bv)d\bv\;.
\end{equation}

First three velocity moments corresponding to three physical values of mass, momentum and energy are written as 
\begin{equation}
  \begin{split}
    \rho_\alpha &= m_\alpha n_\alpha = \int m_\alpha f_\alpha(\bv)d\bv\;,\\
    \rho_\alpha \bu_\alpha &= \int m_\alpha\bv f_\alpha(\bv)d\bv\;,\\
    \frac{D}{2}n_\alpha T_\alpha &= \int \frac{m_\alpha c^2}{2}f_\alpha(\bv)d\bv\;,
  \end{split}
\end{equation}
where $\bc=\bv-\bu_\alpha$, $D=2$ for two dimensional and $D=3$ for three dimensional systems. These moments are mass density $\rho_\alpha$,
momentum density $\rho_\alpha\bu_\alpha$ and granular temperature $T_\alpha$ correspondingly. 

Let us now define the macroparameters for the whole system as
\begin{equation}
  \begin{split}
    \rho &= \sum_{\alpha=1}^{s}\chi_\alpha\cdot\rho_\alpha\;,\\
    \rho\bu &= \sum_{\alpha=1}^{s}\chi_\alpha\cdot\rho_\alpha\bu_\alpha\;,\\
    nT &= \sum_{\alpha=1}^{s}\chi_\alpha\cdot n_\alpha T_\alpha\;,
  \end{split}
\end{equation}
where
\begin{equation}
  n = \sum_{\alpha=1}^{s} n_\alpha = \frac{N}{V}\;.
\end{equation}

If we make the assumption that our system is very large and 
highly non-uniform, i.e. we consider the limits $N\to\infty$ and $s\to\infty$, hence 
\begin{equation}
  \chi_\alpha\rightarrow\chi(\alpha)\;,
\end{equation}
our number density becomes the size distribution function, and (\ref{eq:macroparameter_discrete}) turns into integration over
distribution function
\begin{equation}
  P = \int\limits_1^\infty P(\alpha)\chi(\alpha)d\alpha=\int P(\alpha)d\chi(\alpha)\;,
\end{equation}
and $P_\alpha\rightarrow P(\alpha)$ becomes the function of $\alpha$.

\section{Contact mechanics}

The evolution of the above defined distribution function is governed by Boltzmann equation. In order to write
Boltzmann equation, we need to make sure that our system obeys certain conditions. First is the condition of low density, hence 
the assumption of binary collisions. We assume, that at any time there are only binary interactions take place, 
and never triple, quadruple etc. This means that when we describe the contact mechanics of particles, we consider only two particle
interactions. Second is the molecular chaos assumption, or as Boltzmann described himself, \emph{Stosszahlansatz}. The idea is 
that low density and high amount of particles allows us to assume that they are uncorrelated. Hence, the two-particle distribution
function can be split into two one-particle distribution functions
\begin{equation}
  f_\ab(\bv_\alpha, \bv_\beta)=f_\alpha(\bv_\alpha)\cdot f_\beta(\bv_\beta)\;.
\end{equation}

These assumptions allow us to write the evolution of distribution function as a Boltzmann equation. But first, let us analyze the 
contact mechanics of binary collisions. After a collision between a particle from species $\alpha$ and a particle from species $\beta$,
the change of velocities is written as next
\begin{equation}
  \begin{split}
    \bv'_\alpha &= \bv_\alpha - \frac{\mu}{m_\alpha}(1+\eps)(\bg\cdot\bn)\bn\;,\\
    \bv'_\beta &= \bv_\beta + \frac{\mu}{m_\beta}(1+\eps)(\bg\cdot\bn)\bn\;,
  \end{split}
\end{equation}
where $\bv'_\alpha,\,\bv'_\beta$ are postcollisional velocities, $\bg=\bv_\alpha-\bv_\beta$ is the impact velocity, $\bn$ is the collision
unit vector, pointing from the center of particle $\alpha$ to the center of particle $\beta$ and $\eps$ is the restitution coefficient.
\begin{equation}
  \mu = \frac{m_\alpha m_\beta}{m_\alpha + m_\beta}\;,
\end{equation}
is the effective mass of the collision. Now we can calculate the change of kinetic energy due to collision 
\begin{equation}
  \begin{split}
    \delta E_\alpha &= -\mu(1+\eps)(\bg\cdot\bn)(\bv_\alpha\cdot\bn)+\frac{\mu^2}{2m_\alpha}(1+\eps)^2(\bg\cdot\bn)^2\;,\\
    \delta E_\beta &= +\mu(1+\eps)(\bg\cdot\bn)(\bv_\beta\cdot\bn)+\frac{\mu^2}{2m_\beta}(1+\eps)^2(\bg\cdot\bn)^2\;,
  \end{split}
\end{equation}
or switching into center of mass frame of reference
\begin{equation}
  \begin{split}
    \delta E_\alpha &= -\mu(1+\eps)(\bg\cdot\bn)(\bv_C\cdot\bn)-\frac{1-\eps^2}{2}\frac{\mu^2}{m_\alpha}(\bg\cdot\bn)^2\;,\\
    \delta E_\beta &= +\mu(1+\eps)(\bg\cdot\bn)(\bv_C\cdot\bn)-\frac{1-\eps^2}{2}\frac{\mu^2}{m_\beta}(\bg\cdot\bn)^2\;,
  \end{split}
\end{equation}
where $(m_\alpha+m_\beta)\bv_C=m_\alpha\bv_\alpha+m_\beta\bv_\beta$ is the center of mass velocity. The first terms in both expressions
are identical with opposite signs. It means, that this part of energy is exchanged between the particles and stays in the system, without
being dissipated. The second terms are always negative and describe the amount of energy which is dissipated due to collision. We can see 
that particles with different mass dissipate different amount of energy, or to be more precise, particles with less mass dissipated
more energy. This fact leads to important consequences, such us breakage of energy equipartition and onset of non-identical granular
temperatures of different species. 

\section{Granular temperature decay rate}
The Boltzmann equation for our system is written in the next form 
\begin{equation}\label{eq:Boltzmann_eq}
  \frac{\pd f_\alpha}{\pd t}+\bv_\alpha\frac{\pd f_\alpha}{\pd\bx}-\frac{1}{m_\alpha}\frac{\pd\Phi(\bx)}{\pd\bx}\frac{\pd f_\alpha}{\pd\bv_\alpha}=
  \sum_{\beta=1}^{s}\chi_\beta\cdot I_\ab(\bv_\alpha,\bv_\beta\vert f_\alpha,f_\beta)\;,
\end{equation}
where $\Phi(\bx)$ is the potential function of the external forcing and $I_\ab$ is the collision integral. In this work we are not going to 
solve the Boltzmann equations, but rather assume that the solution of it is given in the Maxwellian form. The equation itself will be used
to derive the hydrodynamic equations for macroparameters. Mainly we will focus on the temperature evolution equations. In order to derive
the temperature evolution equations, we need to multiply the Boltzmann equation by $m_\alpha v_\alpha^2/2$ and integrate over the whole 
velocity space. The left side of the equation will describe the evolution of the temperature due to dynamic flows and external heating. We will
concentrate on it further in our work. At the moment let us describe the temperature decay rate due to pure collisions, which is governed
by the right side of the Boltzmann equation. After integration over the velocity space, the right hand side of the temperature evolution 
equation can be written in the next form
\begin{equation}
  \left\langle\frac{dE_\alpha}{dt}\right\rangle = \sum_{\beta=1}^{s}\chi_\beta\cdot\left\langle\frac{d E_\alpha}{dt}\right\rangle_\beta\;.
\end{equation}

The collision integral has the next property: given a certain dynamic function $\psi(\bv_\alpha)$, the next is true for two dimensional system
\begin{equation}  
  \left\langle\frac{d\psi_\alpha}{dt}\right\rangle=g_2(\sigma)\sigma\int d\bv_\alpha d\bv_\beta\int d\bn\,\Theta(-\bg\cdot\bn)\vert\bg\cdot\bn\vert
  \times f_\alpha(\bv_\alpha)f_\beta(\bv_\beta)\Delta\psi(\bv_\alpha)\;,
\end{equation}
where $g_2(\sigma)=1$ is the pair correlation function, which we assume to unity for simplicity, and $\sigma=\sigma_\alpha+\sigma_\beta$ is 
the cross section of the binary collision, and for two dimensional system it is simply the sum of particles' radii. If we use 
$\psi(\bv_\alpha)=E_\alpha$, then
\begin{equation}
  \Delta\psi(\bv_\alpha)=\delta E_\alpha=-\mu(1+\eps)(\bg\cdot\bn)(\bc_\alpha\cdot\bn)+\frac{\mu^2}{2m_\alpha}(1+\eps)^2(\bg\cdot\bn)^2
  -\mu(1+\eps)(\bg\cdot\bn)(\bu_\alpha\cdot\bn)\;,
\end{equation}
where $\bc_\alpha=\bv_\alpha-\bu_\alpha$. Since $\bg=\bv_\alpha-\bv_\beta=\bc_\alpha-\bc_\beta$, we can change variables as 
$d\bv_\alpha d\bv_\beta=d\bc_\alpha d\bg$. Now, let us introduce several angular variables. First of all, since the collisions are 
isotropic, for two dimensional case we have $d\bn=d\phi$, where we can take $\phi$ being an angle between $\bn$ and stationary vector 
$\bu_\alpha$, and obviously $\phi\in[0, 2\pi]$. The angle between $\bg$ and $\bn$ will be $\theta$ and the angle between $\bg$ and $\bc_\alpha$
will be $\gamma$. Now, we have
\begin{equation}
  \begin{split}
    d\bn &= d\phi\;,\,\,\,\phi\in[0, 2\pi]\;,\\
    \Theta(-\bg\cdot\bn)\vert\bg\cdot\bn\vert &= g\cos\theta\;,\,\,\,\theta\in\left[-\frac{\pi}{2},\frac{\pi}{2}\right]\;,\\
    \bg\cdot\bc_\alpha&= gc_\alpha\cos\gamma\;,\,\,\,\gamma\in[0, 2\pi]\;,\\
    \bu_\alpha\cdot\bn &= u_\alpha\cos\phi\;,\\
    \bc_\alpha\cdot\bn &= c_\alpha\cos(\gamma-\theta)\;,\\
    d\bv_\alpha d\bv_\beta&=d\bc_\alpha d\bg = gc_\alpha dg dc_\alpha d\theta d\gamma\;,
  \end{split}
\end{equation}
and now we can write
\begin{equation}\label{eq:delta_E}
  \delta E_\alpha=-\mu(1+\eps)gc_\alpha\cos\theta\cos(\gamma-\theta)+\frac{\mu^2}{2m_\alpha}(1+\eps)^2g^2\cos^2\theta
  -\mu(1+\eps)gu_\alpha\cos\theta\cos\phi\;.
\end{equation}

To make further progress, we need to have a knowledge about the distribution function. The simplest case is the Maxwellian function, which
we will be using throughout the course of our work. Hence, for two dimensional case we have
\begin{equation}
  f_\alpha(\bv_\alpha)=\frac{n_\alpha m_\alpha}{2\pi T_\alpha}\exp\left\{-\frac{m_\alpha(\bv_\alpha-\bu_\alpha)^2}{2T_\alpha}\right\}\;,
\end{equation}
or 
\begin{equation}
  f_\alpha(\bc_\alpha)=\frac{n_\alpha\kp_\alpha}{\pi}\exp\left(-\kp_\alpha c_\alpha^2\right)\;,
\end{equation}
where
\begin{equation}
  \kp_\alpha = \frac{m_\alpha}{2T_\alpha}\;.
\end{equation}
Now we can write
\begin{equation}
  f_\alpha(\bc_\alpha)f_\beta(\bc_\alpha-\bg)=\frac{n_\alpha n_\beta\kp_\alpha\kp_\beta}{\pi^2}
  \exp\left(-(\kp_\alpha+\kp_\beta)c_\alpha^2-\kp_\beta g^2+2\kp_\beta g c_\alpha\cos\gamma\right)\;.
\end{equation}
Since, the angle $\phi$ appears only in the last term of (\ref{eq:delta_E}), we can perform the integration over $\phi$ immediately.
The last term depending on the mean flow velocity vanishes, as 
\begin{equation}
  \int\limits_0^{2\pi}\cos\phi\,d\phi=0\;,
\end{equation}
and the other terms will have a factor of $2\pi$. Now, combining all terms and substitutions, we can finally write
\begin{equation}
  \begin{split}
    \left\langle\frac{dE_\alpha}{dt}\right\rangle_\beta = &\nu_\ab\int dg dc_\alpha\,g^2c_\alpha
    \exp(-(\kp_\alpha+\kp_\beta)c_\alpha^2-\kp_\beta g^2)\int\limits_0^{2\pi}d\gamma\exp(R\cos\gamma)\int\limits_{-\pi/2}^{\pi/2}d\theta\cos\theta\times\\
    &\times\left[-\mu(1+\eps)gc_\alpha\cos\theta\cos(\gamma-\theta)+\frac{\mu^2}{2m_\alpha}(1+\eps)^2g^2\cos^2\theta\right]\;,
  \end{split}
\end{equation}
where
\begin{equation}
  \begin{split}
    \nu_\ab &= \frac{2\sigma}{\pi}n_\alpha n_\beta\kp_\alpha\kp_\beta\;,\\
    R &= 2\kp_\beta gc_\alpha\;.
  \end{split}
\end{equation}
Splitting $\cos(\gamma-\theta)=\cos\gamma\cos\theta+\sin\gamma\sin\theta$, we have
\begin{equation}
  \begin{split}
    \left\langle\frac{dE_\alpha}{dt}\right\rangle_\beta = &\nu_\ab\int dg dc_\alpha\,g^2c_\alpha
    \exp(-(\kp_\alpha+\kp_\beta)c_\alpha^2-\kp_\beta g^2)\int\limits_0^{2\pi}d\gamma\exp(R\cos\gamma)\times\\
    &\times\left[-\mu(1+\eps)gc_\alpha\cdot(\cos\gamma\cdot S_{\theta,2}+\sin\gamma\cdot S_{\theta,1})
    +\frac{\mu^2}{2m_\alpha}(1+\eps)^2g^2\cdot S_{\theta,2}\right]\;,
  \end{split}
\end{equation}
where
\begin{equation}
  \begin{split}    
    S_{\theta,1} &= \int\limits_{-\pi/2}^{\pi/2}\cos^2\theta\sin\theta\,d\theta=0\;,\\
    S_{\theta,2} &= \int\limits_{-\pi/2}^{\pi/2}\cos^3\theta\,d\theta=\frac{4}{3}\;.
  \end{split}
\end{equation}
Now we have
\begin{equation}
  \begin{split}
    \left\langle\frac{dE_\alpha}{dt}\right\rangle_\beta = &\nu_\ab\int dg dc_\alpha\,g^2c_\alpha
    \exp(-(\kp_\alpha+\kp_\beta)c_\alpha^2-\kp_\beta g^2)\int\limits_0^{2\pi}d\gamma\exp(R\cos\gamma)\times\\
    &\times\left[\frac{2\mu^2}{3m_\alpha}(1+\eps)^2g^2-\frac{4\mu}{3}(1+\eps)gc_\alpha\cos\gamma\right]\;,
  \end{split}
\end{equation}
and the integration over $\gamma$ becomes
\begin{equation}
  \begin{split}
    \left\langle\frac{dE_\alpha}{dt}\right\rangle_\beta = &\nu_\ab\int dg dc_\alpha\,g^2c_\alpha
    \exp(-(\kp_\alpha+\kp_\beta)c_\alpha^2-\kp_\beta g^2)\times\\
    &\times\left[\frac{2\mu^2}{3m_\alpha}(1+\eps)^2g^2\cdot S_{\gamma,1}
    -\frac{4\mu}{3}(1+\eps)gc_\alpha\cdot S_{\gamma,2}\right]\;,
  \end{split}
\end{equation}
where
\begin{equation}
  \begin{split}
    S_{\gamma,1} &= \int\limits_{0}^{2\pi}\exp(R\cos\gamma)\,d\gamma=2\pi I_0(R)\;,\\
    S_{\gamma,2} &= \int\limits_{0}^{2\pi}\cos\gamma\exp(R\cos\gamma)\,d\gamma=2\pi I_1(R)\;,
  \end{split}
\end{equation}
where $I_\nu(R),\,\nu=0,1$ are the modified Bessel functions. Now we can write
\begin{equation}
  \begin{split}
    \left\langle\frac{dE_\alpha}{dt}\right\rangle_\beta = &\frac{4\pi}{3}\nu_\ab\int\limits_{0}^{\infty}dg\,g^2\exp(-\kp_\beta g^2)
    \int\limits_{0}^{\infty}dc_\alpha\,c_\alpha\exp(-pc_\alpha^2)\times\\
    &\times\left[\frac{\mu^2}{m_\alpha}(1+\eps)^2g^2\cdot I_0(rc_\alpha)-
    2\mu(1+\eps)gc_\alpha\cdot I_1(rc_\alpha)\right]\;,
  \end{split}
\end{equation}
where $p=\kp_\alpha+\kp_\beta$ and $r=2\kp_\beta g$. Using the next properties of the Bessel functions 
\begin{equation}
  \begin{split}
    \int\limits_{0}^{\infty}dc_\alpha\,c_\alpha\exp(-pc_\alpha^2)I_0(rc_\alpha)&=\frac{1}{2p}\exp\left(\frac{r^2}{4p}\right)\;,\\
    \int\limits_{0}^{\infty}dc_\alpha\,c_\alpha^2\exp(-pc_\alpha^2)I_1(rc_\alpha)&=\frac{r}{4p^2}\exp\left(\frac{r^2}{4p}\right)\;,
  \end{split}
\end{equation}
and performing the standard integration
\begin{equation}
  \int\limits_{0}^{\infty}g^4e^{-\lambda g^2}\,dg =\frac{8}{3\lambda^2}\sqrt{\frac{\pi}{\lambda}}\;,\,\,\,\lambda>0\;,
\end{equation}
we end up with 
\begin{equation}
  \left\langle\frac{dE_\alpha}{dt}\right\rangle_\beta = \sqrt{\frac{\pi}{\lambda}}\cdot\sigma n_\alpha n_\beta\times
  \left[\frac{\mu(1+\eps)}{2m_\alpha\lambda}-\frac{1}{\kp_\alpha}\right]\mu(1+\eps)\;,
\end{equation}
where
\begin{equation}
  \lambda = \frac{\kp_\alpha\kp_\beta}{\kp_\alpha+\kp_\beta}=\frac{1}{2}\frac{m_\alpha m_\beta}{m_\alpha T_\beta+m_\beta T_\alpha}\;.
\end{equation}
After straightforward arithmetic manipulations, we can finally write
\begin{equation}
  \left\langle\frac{dE_\alpha}{dt}\right\rangle_\beta = \sqrt{2\pi}\sigma n_\alpha n_\beta\cdot\frac{\mu}{m_\alpha}
  \sqrt{\frac{m_\alpha T_\beta+m_\beta T_\alpha}{m_\alpha m_\beta}}\left[-(1-\eps^2)T_\alpha + \frac{\mu}{m_\beta}(1+\eps)^2
  (T_\beta-T_\alpha)\right]\;.
\end{equation}
In order to give the physical meaning to this result, let us first calculate the collision frequency $\omega_\ab$ between species $\alpha$
and $\beta$. By simple logic we can write that collision frequency is
\begin{equation}
  \omega_\ab = n_\alpha n_\beta\sigma_\ab\langle g_\ab\rangle\;,
\end{equation}
where $\langle g_\ab\rangle$ is the mean value of the impact velocity. The calculation of any two particle dynamic function can be performed
by the following expression
\begin{equation}
  \langle g_\ab\rangle=\frac{\int g_\ab\cdot f_\alpha(\bc_\alpha)f_\beta(\bc_\beta)\,d\bc_\alpha d\bc_\beta}
  {\int f_\alpha(\bc_\alpha)f_\beta(\bc_\beta)\,d\bc_\alpha d\bc_\beta}\;,
\end{equation}
and using Maxwellian form of distribution function, we have 
\begin{equation}
  \langle g_\ab\rangle=\sqrt{\frac{\pi}{2}\frac{m_\alpha T_\beta+m_\beta T_\alpha}{m_\alpha m_\beta}}\;,
\end{equation}
and write 
\begin{equation}
  \left\langle\frac{dE_\alpha}{dt}\right\rangle_\beta = \omega_\ab\times\left(-(1-\eps^2)\frac{2\mu}{m_\alpha}T_\alpha+
  (1+\eps)^2\frac{2\mu^2}{m_\alpha m_\beta}(T_\beta-T_\alpha)\right)\;.
\end{equation}

\section{Differentially rotating flat disk}
We assume azimuthally symmetric flat disk with rotational frequency 
\begin{equation}
  \Omega(r) = \sqrt{\frac{GM}{r^3}}\;,
\end{equation}
where $G$ is gravitational constant, $M$ is the mass of central planet and $r$ is the distance to the center of the planet.
Switching to the frame of reference, rotating with frequency $\Omega_0=\Omega(r_0)$ at a distance $r_0$ from the center of the planet,
the equation of motion for particles in such a box will be governed by the so called Hill's equations. In two dimensional case and in the 
absence of kicks from moons, they read
\begin{equation}
  \begin{split}
    \ddot{x}-2\Omega_0\dot{y}-3\Omega_0^2x &= 0\;,\\
    \ddot{y}+2\Omega_0\dot{x} &= 0\;,
  \end{split}
\end{equation} 
where $x$ is the radial distance from the center of the box, $y$ is orbital distance from the center of the box, and $x,y\ll r_0$ is assumed.
Since the rotation of the disk is stationary, the tangential acceleration is absent $\ddot{x}=\ddot{y}=0$, hence
\begin{equation}\label{eq:Hill}
  \begin{split}
    \dot{x} &= 0\;,\\
    \dot{y} &= -\frac{3}{2}\Omega_0 x\;,
  \end{split}
\end{equation}
and the mean flow velocity $\bu=\dot{x}\be_x+\dot{y}\be_y$ is
\begin{equation}
  \bu = -\frac{3}{2}\Omega_0 x\cdot\be_y\;,
\end{equation}
where $\be_x$ and $\be_y$ are radial and orbital unit vectors correspondingly. Now, multiplying first and second equations in (\ref{eq:Hill})
by $\dot{x}$ and $\dot{y}$, and summing them, we have
\begin{equation}
  \dot{x}\ddot{x} + \dot{y}\ddot{y} = 3\Omega_0^2x\dot{x}\;,
\end{equation}
and since $v^2=\dot{x}^2+\dot{y}^2$, we can write
\begin{equation}
  \frac{d}{dt}\left(\frac{v^2}{2}\right)=\dot{x}\ddot{x} + \dot{y}\ddot{y} = 3\Omega_0^2x\dot{x} = 
  \frac{d}{dt}\left(\frac{3}{2}\Omega_0^2x^2\right)\;,
\end{equation}
or 
\begin{equation}
  \frac{dH(\bx,\bv)}{dt}=\frac{d}{dt}\left(\frac{mv^2}{2}+\Phi(\bx)\right)=0\;,
\end{equation}
where $\Phi(\bx)$ is the potential function of a particle with mass $m$
\begin{equation}
  \Phi(\bx) = -\frac{3}{2}m\Omega_0^2x^2\;.
\end{equation}
Now, the acceleration of a particle can be obtained from
\begin{equation}
  m\bw = -\frac{\pd\Phi}{\pd x}\be_x-\frac{\pd\Phi}{\pd y}\be_y\;,
\end{equation}
with the result
\begin{equation}
  \bw = 3\Omega_0^2 x\cdot\be_x\;.
\end{equation}
This is the centripetal acceleration in the non-inertial rotating frame of reference.

\section{Hydrodynamic approximation}
The Boltzmann equation (\ref{eq:Boltzmann_eq}) can be used to derive the hydrodynamic equations for certain macroparameter $A(\bv)$,
or the so called balance equations. In order to do so, one needs to multiply (\ref{eq:Boltzmann_eq}) by $A(\bv)$ and integrate
over $\bv$. Then, after simple manipulations we obtain 
\begin{equation}
  \frac{\pd}{\pd t}\int Af_\alpha d\bv + \frac{\pd}{\pd x_i}\int Af_\alpha v_i d\bv
  -\int f\left[\frac{\pd A}{\pd t}+v_i\frac{\pd A}{\pd x_i}+w_i\frac{\pd A}{\pd v_i}\right]d\bv=\left(\frac{\pd A}{\pd t}\right)_c\;,
\end{equation}
where we used the fact that 
\begin{equation}
  \int\frac{\pd}{\pd v_i}(Af_\alpha w_i)d\bv=\oint Af_\alpha w_i\cdot d{\bm{\sigma}} = 0\;.
\end{equation}
Here and further by $i,j,k$ we will denote the cartesian coordinates. Using the next notations
\begin{equation}
  \begin{split}
    n_\alpha &= \int f_\alpha\,d\bv\;,\\
    n_\alpha u_i &= \int v_i f_\alpha\,d\bv\;,\\
    p_{ijk} &= \int (v_i-u_i)(v_j-u_j)(v_k-u_k)f_\alpha\,d\bv\;,
  \end{split}
\end{equation}
where $\bu_\alpha=\bu=u_i$ is the mean flow velocity and $p_{ijk}$ is the pressure tensor, and consequently setting $A=1$, $A=v_i$ and $A=v_iv_j$
we will obtain continuity equation, momentum balance equation and viscous-stress equation. Let us first write the next macroparameters
\begin{equation}
  \begin{split}
    \int v_iv_j f_\alpha\,d\bv &= p_{ij}+n_\alpha u_iu_j\;,\\
    \int v_iv_jv_k f_\alpha\,d\bv &= p_{ijk} + u_ip_{jk} + u_jp_{ik} + u_kp_{ij} + n_\alpha u_iu_ju_k\;,
  \end{split}
\end{equation}
and using 
\begin{equation}
  w_i = -\frac{1}{m_\alpha}\frac{\pd\Phi}{\pd x_i}=-\frac{\pd U}{\pd x_i}\;,
\end{equation}
we write
\begin{equation}
  \begin{split}
    \frac{\pd n_\alpha}{\pd t} + \frac{\pd}{\pd x_i}(n_\alpha u_i) &= \left(\frac{\pd n_\alpha}{\pd t}\right)_c\;,\\
    \frac{\pd}{\pd t}(n_\alpha u_i) + \frac{\pd}{\pd x_j}(p_{ij}+n_\alpha u_iu_j)+n_\alpha\frac{\pd U}{\pd x_i} 
    &= \left(\frac{\pd}{\pd t}(n_\alpha u_i)\right)_c\;,\\
    \frac{\pd}{\pd t}(p_{ij}+n_\alpha u_iu_j)+n_\alpha\left(u_i\frac{\pd U}{\pd x_j}+u_j\frac{\pd U}{\pd x_i}\right) +& \\
    +\frac{\pd}{\pd x_k}(p_{ijk}+u_ip_{jk}+u_jp_{ik}+u_kp_{ij}+n_\alpha u_iu_ju_k) &= \left(\frac{\pd}{\pd t}(p_{ij}+n_\alpha u_iu_j)\right)_c\;,
  \end{split}
\end{equation}
and after some algebra we obtain
\begin{equation}
  \begin{split}
    \frac{\pd n_\alpha}{\pd t} + \frac{\pd}{\pd x_i}(n_\alpha x_i) &= 0\;,\\
    \frac{\pd u_i}{\pd t} + u_j\frac{\pd u_i}{\pd x_j} + \frac{1}{n_\alpha}\frac{\pd p_{ij}}{\pd x_j} &= -\frac{\pd U}{\pd x_i}\;,\\
    \frac{\pd p_{ij}}{\pd t}+p_{jk}\frac{\pd u_i}{\pd x_k} + p_{ik}\frac{\pd u_j}{\pd x_k} + \frac{\pd}{\pd x_k}(u_kp_{ij}) 
    &= \left(\frac{\pd p_{ij}}{\pd t}\right)_c\;,
  \end{split}
\end{equation}
where we made the next assumptions
\begin{equation}
  \begin{split}
    \left(\frac{\pd n_\alpha}{\pd t}\right)_c &= 0\;,\\
    \left(\frac{\pd}{\pd t}(n_\alpha u_i)\right)_c &= 0\;,\\
    \frac{\pd p_{ijk}}{\pd x_k} &= 0\;,
  \end{split}
\end{equation}
because of mass and momentum conservation. Also, we assumed that the random motion $\pd p_{ijk}/\pd x_k$ is much smaller than the mean flow
motion, hence was neglected.







%  \cite{Brilliantov:2004book, Bodrova:2014epl_steep_distr, Brilliantov:2007pre_coll_dyn, 
%  Brilliantov:2007pre_coll_adh, Schwager:2007gm_coll_dyn, Dilley:1993icarus_energy_loss, 
%  Garzo:2012pre_maxwell_gas, DeSoria:2013pre_hydro_gas, Schaefer:1996jphys_force_schemes, 
%  Garzo:2007pre_enskog_I, Garzo:2007pre_enskog_II, Garzo:1999pre_gran_mixture, 
%  Geminard:2004pre_gran_pressure, Quinn:2010astro_hill_simplectic, Barrat:2002gm_binary_mix, 
%  Hoffmann:2013astrolett_vertical_relax, Cuendet:2007jchem_md_simul, Uecker:2009pre_part_energy, 
%  Morishima:2006icarus_dense_ring_simul, Ohtsuki:1998icarus_vel_disp, Salo:2010icarus_N_body, 
%  Greenberg:1988icarus, Spahn:2006gamm_hydro_rings, Lois:2007pre_shear_flow, 
%  Spahn:2004euro_lett_kinetic_fraggr, Spahn:2000icarus_stability_analysis, WisdomTremaine:1988astro}.


  \bibliographystyle{apsrev4-1}
  \bibliography{shearedgas.bib}

\end{document}
