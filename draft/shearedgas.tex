\documentclass[preprint, aps, pra]{revtex4-1}

\usepackage{amsmath}
\usepackage{amssymb}

%\usepackage[default]{sourcesanspro}
%\usepackage[T1]{fontenc}

\newcommand{\ab}{\alpha\beta}
\newcommand{\eps}{\varepsilon}
\newcommand{\bx}{{\bf{x}}}
\newcommand{\bv}{{\bf{v}}}
\newcommand{\bu}{{\bf{u}}}
\newcommand{\bc}{{\bf{c}}}

\begin{document}
  \title{Sheared Granular Gas Dynamics}
  \author{Yernur Baibolatov}
  \affiliation{Universit\"{a}t Potsdam, Institut f\"{u}r Physik und Astronomie}
  \maketitle
  
\section{Non-uniform granular gas system}
 In this work we consider a rotating disk of granular gases with non-uniform size distribution of constituents
 under external gravitational shear. Granular gases are well known for its intrinsic dissipative nature and hence
 decay of velocity dispersion or \emph{granular temperature} if the system has nonzero initial energy. Let us start
 our work with notations and definitions of necessary parameters. The total number of constituents in the system is $N$.
 The number of species in our system is $s$ and the total number of constituents of species $\alpha$ is $N_\alpha$, 
 where $\alpha\in[1,s]$.
 Hence, we can write
 \begin{equation}
   \chi_\alpha=\frac{N_\alpha}{N}\;,\,\,\,\sum_{\alpha=1}^{s}\chi_\alpha=1\;,
 \end{equation}
 where $\chi_\alpha$ is the concentration of constituents of species $\alpha$.
This means that there are $N_\alpha$ identical constituents with masses $m_\alpha$. 
Obviously $m_\alpha\neq m_\beta$ if $\alpha\neq\beta$. Since a system of certain species consists of large number
of constituents $N_\alpha$, it is considered as a statistical system and we analyze it through certain macroscopic parameters.
Let us say that $P_\alpha$ is one of the macroscopic parameters of species $\alpha$, then for the whole system we can write
\begin{equation}\label{eq:macroparameter_discrete}
  P = \sum_{\alpha=1}^{s}\chi_\alpha\cdot P_\alpha\;.
\end{equation}
This is the mean value of macroparameter for the whole system across all species.

In order to define the macroparameter $P_\alpha$ itself, we need to introduce the one-particle distribution function in the phase
space of dynamic variables. The only dynamic variables of a single particle are coordinate $\bx$ and velocity $\bv$.
Now, the one-particle distribution function, or simply distribution function, for species $\alpha$ is written as $F_{\alpha}(\bx,\bv)$.
This function should have the next property
\begin{equation}
  \int F_\alpha(\bx,\bv)d\bx d\bv = n_\alpha = \frac{N_\alpha}{V}\;,
\end{equation}
where the integration is performed over all phase space and $n_\alpha$ is the number density of species $\alpha$, $V$ is the total volume 
of the system. Note that $n_\alpha$ here is the average number density across the whole system volume $V$. We can write separately 
coordinate and velocity distribution functions as 
\begin{equation}
  \begin{split}
    g_\alpha(\bx) &= \int F_\alpha(\bx,\bv)d\bv\;,\\
    f_\alpha(\bv) &= \int F_\alpha(\bx,\bv)d\bx\;,
  \end{split}
\end{equation}
and obviously
\begin{equation}
  \int g_\alpha(\bx)d\bx = \int f_\alpha(\bv)d\bv = n_\alpha\;.
\end{equation}
Further, we will mostly use the velocity distribution function, since we know that all informative
macroscopic parameters describing statistical systems are certain moments of dynamic functions of velocity $p_\alpha=p_\alpha(\bv)$.
Hence corresponding macroparameter $P_\alpha$ is obtained from
\begin{equation}
  n_\alpha P_\alpha = \int p_\alpha(\bv)f_\alpha(\bv)d\bv\;.
\end{equation}

First three velocity moments corresponding to three physical values of mass, momentum and energy are written as 
\begin{equation}
  \begin{split}
    \rho_\alpha &= m_\alpha n_\alpha = \int m_\alpha f_\alpha(\bv)d\bv\;,\\
    \rho_\alpha \bu_\alpha &= \int m_\alpha\bv f_\alpha(\bv)d\bv\;,\\
    \frac{D}{2}n_\alpha T_\alpha &= \int \frac{m_\alpha c^2}{2}f_\alpha(\bv)d\bv\;,
  \end{split}
\end{equation}
where $\bc=\bv-\bu_\alpha$, $D=2$ for two dimensional and $D=3$ for three dimensional systems. These moments are mass density $\rho_\alpha$,
momentum density $\rho_\alpha\bu_\alpha$ and granular temperature $T_\alpha$ correspondingly. 

Let us now define the macroparameters for the whole system as
\begin{equation}
  \begin{split}
    \rho &= \sum_{\alpha=1}^{s}\chi_\alpha\cdot\rho_\alpha\;,\\
    \rho\bu &= \sum_{\alpha=1}^{s}\chi_\alpha\cdot\rho_\alpha\bu_\alpha\;,\\
    nT &= \sum_{\alpha=1}^{s}\chi_\alpha\cdot n_\alpha T_\alpha\;,
  \end{split}
\end{equation}
where
\begin{equation}
  n = \sum_{\alpha=1}^{s} n_\alpha = \frac{N}{V}\;.
\end{equation}

If we make the assumption that our system is very large and 
highly non-uniform, i.e. we consider the limits $N\to\infty$ and $s\to\infty$, hence 
\begin{equation}
  \chi_\alpha\rightarrow\chi(\alpha)\;,
\end{equation}
our number density becomes the size distribution function, and (\ref{eq:macroparameter_discrete}) turns into integration over
distribution function
\begin{equation}
  P = \int\limits_1^\infty P(\alpha)\chi(\alpha)d\alpha=\int P(\alpha)d\chi(\alpha)\;,
\end{equation}
and $P_\alpha\rightarrow P(\alpha)$ becomes the function of $\alpha$.


  \cite{Brilliantov:2004book, Bodrova:2014epl_steep_distr, Brilliantov:2007pre_coll_dyn, 
  Brilliantov:2007pre_coll_adh, Schwager:2007gm_coll_dyn, Dilley:1993icarus_energy_loss, 
  Garzo:2012pre_maxwell_gas, DeSoria:2013pre_hydro_gas, Schaefer:1996jphys_force_schemes, 
  Garzo:2007pre_enskog_I, Garzo:2007pre_enskog_II, Garzo:1999pre_gran_mixture, 
  Geminard:2004pre_gran_pressure, Quinn:2010astro_hill_simplectic, Barrat:2002gm_binary_mix, 
  Hoffmann:2013astrolett_vertical_relax, Cuendet:2007jchem_md_simul, Uecker:2009pre_part_energy, 
  Morishima:2006icarus_dense_ring_simul, Ohtsuki:1998icarus_vel_disp, Salo:2010icarus_N_body, 
  Greenberg:1988icarus, Spahn:2006gamm_hydro_rings, Lois:2007pre_shear_flow, 
  Spahn:2004euro_lett_kinetic_fraggr, Spahn:2000icarus_stability_analysis, WisdomTremaine:1988astro}.


  \bibliographystyle{apsrev4-1}
  \bibliography{shearedgas.bib}

\end{document}
